% This is samplepaper.tex, a sample chapter demonstrating the
% LLNCS macro package for Springer Computer Science proceedings;
% Version 2.20 of 2017/10/04
%
\documentclass[runningheads]{llncs}
%
\usepackage{graphicx}
\usepackage[utf8]{inputenc}
\usepackage{subfig}
% Used for displaying a sample figure. If possible, figure files should
% be included in EPS format.
%
% If you use the hyperref package, please uncomment the following line
% to display URLs in blue roman font according to Springer's eBook style:
% \renewcommand\UrlFont{\color{blue}\rmfamily}

\begin{document}
%
\title{Visual Analytics of Large Bipartite Networks assisted by Multilevel Strategies\thanks{Supported by FAPESP, project 2017/05838-3}}
%
\titlerunning{Multilevel Visualization of Bipartite Networks}
% If the paper title is too long for the running head, you can set
% an abbreviated paper title here
%
\author{Renato Fabbri\orcidID{0000-0002-9699-629X} \and
Alan Valejo \and
Maria Cristina Ferreira de Oliveira\orcidID{0000-0002-4729-5104}}
%
\authorrunning{R. Fabbri et al.}
% First names are abbreviated in the running head.
% If there are more than two authors, 'et al.' is used.
%
\institute{University of São Paulo, São Carlos SP, BR\\
\email{renato.fabbri@gmail.com},
\email{alanvalejo@gmail.com},
\email{cristina@icmc.usp.br}\\
\url{http://conteudo.icmc.usp.br/pessoas/cristina/}}
%
\maketitle
%
\begin{abstract}
% The abstract should briefly summarize the contents of the paper in
% 150--250 words.
  It is a well established fact that bipartite, or two-layer, networks are pervasive
  in model real-world phenomena and that they play fundamental roles in
  graph theory.
  Multilevel strategies have been developed for optimization tasks,
  and for the visualization of simple (``unipartite'') networks,
  but their employment for visualizing bipartite networks were not found by the authors.
  In this work, we present advances in the use of multilevel strategies for the
  visualization of bipartite networks,
  allowing interactive and intuitive navigation of such structures and visual mappings of large datasets.
  More specifically, we developed a visual analytics web interface in which
  a parametrizable simpification of
  bipartite networks are obtained through the application of coarsening algorithms.
  The resulting networks are then presented to the user,
  providing a genuine route for the ``overview first - focus on demand''
  process on the analysis of the undelying data, in which the analyst
  selects supervertices or whole network sectors for more detailed observation,
  i.e. performs requests for the interface to display 
  specific structures in less simplified settings.
  Moreover, the application is useful for the development multilevel strategies
  e.g. by the specification of vertices to guide the coarsening processes and
  the examination of the resulting multilevel hierarchy.

\keywords{Network visualization \and Multilevel strategies \and Visual analytics \and Big data \and Complex networks \and Data visualization.}
\end{abstract}
%
\section{Introduction}
The visualization of large-scale networks poses challenges both in terms of computational costs
and of effective presentation of the information for the user~\cite{tang,staudt}.
These issues may be aggravated in the case of bipartite networks,
due to their sparsity and topological complexities~\cite{alan1}.
Bipartite networks are comprised of two partitions of nodes,
called ``layers'',
and links are not incident between nodes in the same partition.
Such network type arises very often and naturally from the representation
of relations among two kinds objects,
e.g. documents and terms or authors~\cite{doc,sci,movie}, or patient and gene~\cite{gene}.
Furthermore, real-world networks are often bipartite, and most unipartite networks
are projections of bipartite networks or may be considered as exibiting bipartite properties~\cite{guillaume0,guillaume}.
In order to assist the visualization and navigation of large networks, one possibility is the use of
multilevel strategies, which consist on the employment of incremental coarsening of the original
network to obtain a sequence of simplified representations.
Multilevel strategies are most traditionally used for executing complex algorithms
on large-scale networks by the application of the algorithm on a smaller
version of the network~\cite{alan2,ml2}.
Their employment of multilevel strategies for the visualization of simple (i.e. ``unipartite'')
networks have been reported, but their exploitation for visualizing bipartite networks
was not found by the authors.
Accordingly, we present a system for the visualization of bipartite networks using
multilevel strategies developed for bipartite networks.
The system consists on presented a simplified version of the network for the user, which then
requests for supervertices (or collections of them) to be uncoarsened and presented in more detail.

This paper is organized as follows: in Section~\ref{rel} the related work is examined, while in~\ref{nom} are selected remarks about the vocabulary.
Fundamental concepts are introduced in Section~\ref{bac}, and the method is delineated in Section~\ref{des}.
The software implementation is then described in Section~\ref{sof}.
Results and discussion are in~\ref{res}.
Finally, Section~\ref{con} holds concluding and further work statements.

\subsection{Related work}\label{rel}
%%%
% artigos que devemos citar pelo histórico científico
% da estratégia ML, da visualização ML
% e os que devemos citar pelo ML em redes bipartidas
% e desenvolvidos no ICMC
Multilevel strategies have been employed to visualize unipartite networks~\cite{u1,u2,u3,u4,u5,u6,u7}.
Also, the aggregation of clusters have been reported, and comprises an approach that resembles
the coarsening procedure in creating simplified representations of the original network~\cite{a1,a2,a3,a4,a5,a6}.
Even so, the authors are not aware of previous reports on the use of multilevel strategies
for the visualization and navigation of bipartite networks.



\subsection{Nomenclature and conceptual remarks}\label{nom}
%%%
% level vs layer
% simple and heterogeneous graphs
% multiplex, ?
% multilevel strategies for optimization
% supernode metanode, 

\section{Backgroud}\label{bac}
\section{Method description}\label{des}
%%%
% mlpb and other methods developed
% arguments for the preferred method (mlpb?)
\begin{figure}[!h]\centering
% \includegraphics[width=.7\textwidth]{}
  \caption{.
  }\label{fig:glob}
\end{figure}

\noindent 
\section{Software implementation}\label{sof}
%%%
% description of the functionalities/tools implemented,
% of the technologies and libraries used
% optimization of the computational capacity for large networks by
% means of Web GL, triangles for nodes and details on demand

\section{Results and discussion}\label{res}
%%%
% consistent use of overview first, details on demand
% a first contribution on ML vis. of bipartite networks
% simple interactivity strategies

\section{Conclusions and further work}\label{con}
%%%
% extension to multipartite (or heterogeneous) networks.
% use for marking nodes as pivots for coarsening

\bibliographystyle{splncs04}
\bibliography{biblio}
%
% \begin{thebibliography}{8}
% \bibitem{ref_article1}
% Author, F.: Article title. Journal \textbf{2}(5), 99--110 (2016)
% 
% \bibitem{ref_lncs1}
% Author, F., Author, S.: Title of a proceedings paper. In: Editor,
% F., Editor, S. (eds.) CONFERENCE 2016, LNCS, vol. 9999, pp. 1--13.
% Springer, Heidelberg (2016). \doi{10.10007/1234567890}
% 
% \bibitem{ref_book1}
% Author, F., Author, S., Author, T.: Book title. 2nd edn. Publisher,
% Location (1999)
% 
% \bibitem{ref_proc1}
% Author, A.-B.: Contribution title. In: 9th International Proceedings
% on Proceedings, pp. 1--2. Publisher, Location (2010)
% 
% \bibitem{ref_url1}
% LNCS Homepage, \url{http://www.springer.com/lncs}. Last accessed 4
% Oct 2017
% \end{thebibliography}
\end{document}
